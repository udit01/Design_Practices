\documentclass[12pt]{report}

\usepackage{amssymb,amsthm}
\usepackage{amsmath,color}
%\usepackage{dsfont}
\usepackage{setspace}
\usepackage{graphicx}
\usepackage{multicol}
\usepackage[document]{ragged2e}


\def\w{\mathtt w}
\def\K{\mathcal K}
\def\N{\mathbb N}
\def\R{\mathcal R}
\def\l{\ell}
\def\S{\mathcal S}
\def\Z{\mathbb Z}
\def\ora{\overrightarrow}
\newtheorem{thm}{Theorem}[chapter]
\newtheorem{defn}{Definition}[chapter]

\begin{document}

\begin{center}
\LARGE{\textbf{Indian Institute of Technology, Delhi}}\\
\vspace{1cm}
\large{\textbf{Design Practices in Computer Science}}\\[5pt]
\large{\textbf{COP290}}\\[5pt]
\large{\textbf{Mathematical Model}}\\[5pt]
%by \\
\vspace{0.5cm}

\large{\textbf{Design and Implement a Software }}
\large{\textbf{\\for Engineering Drawing}}\\[5pt]



%\vspace{1in}

\begin{center}
\includegraphics[height=4.5cm]{iitd.jpeg}
\end{center}
\vspace{0.2cm}

\textbf{January 18, 2018} \\
\textbf{Department of Computer Science and Engineering} \\
%\textbf{Indian Institute of Technology, Delhi}\\
%\textbf{16 January 2018}


\vspace{2cm}


\begin{multicols*}{2}

\begin{flushleft}

\textit{Authors :\\ }


\textbf{Udit Jain} \\
(2016CS10327)\\
\textbf{Shashank Goel}
(2016CS10332)\\

\end{flushleft}


\columnbreak

\begin{flushleft}

\textit{\\Supervisor :\\ }
\textbf{Prof. Subhashish Bannerjee} \\[5pt]

\end{flushleft}

\end{multicols*}

\end{center}

\newpage



\begin{center}
\Large \bf ABSTRACT
\end{center}
\vspace{0.2in}

We are going to design and implement a Software Package for Engineering drawing that shall be described and portrayed in a series of five steps to finely work out the design, analysis and the modelling.
\\
\vspace{0.3cm}
The package will have the following functionalities:

\begin{enumerate}
  \item
  We will be able to interactively input or read from a file either
  \begin{itemize}
    \item
    An isometric drawing and a 3D object model 
    \item
    Projections onto any cross section
  \end{itemize}
  \item
  For a given 3D model description, the software will be able to \textit{generate projections} onto any cutting plane or any cross section
  \item
  Given two or more projections, we will be able to interactively \textit{reconstruct the 3D model} of the object and produce the isometric view along any view direction

\end{enumerate}

\vspace{0.5cm}
In this design project, we shall work as developers and algorithm enthusiasts to understand the ways and finding different means to approach and tackle the objectives in a more well defined mathematical way. The solutions shall be presented not completely on how the human brain formulates or understands/interprets a given figure, be it 2D or 3D but in a way, that shall work out in all the cases we deal with in real life and definitely be understandable by the machine. Mathematical explanations that are more amenable to intuition are given.

\\
\vspace{0.2cm}

Being an amateur in this field of design of software to compute projections and reconstruction of the model, it might eventually happen that the algorithm might fail in some cases or it may be proved that such an algorithm cannot exist or the model be correct but be based on certain assumptions on the construction of the object or the projections. Nevertheless, we shall work with full confidence and zeal to achieve the goal or reach to quite an end of the problem so that using our lemmas, proofs and knowledge, someday a perfect model can be implemented using a software by some other Computer Explorer. 
\\
\vspace{0.2cm}

As a matter of interest, we just wish to argue that these things can be computed by our brain so we do hope to find a solution to this problem using machine learning algorithms. Since, Machine Learning algorithms are more or less based on Mathematical matrices, with the use of computer graphics, we expect to find a start with matrices that we have dealt with further in this report. 

\newpage

\tableofcontents

\newpage

\chapter{Introduction}

In the last ten years, a significant progress occurred in the area of 3D graphics. Many studies have been conducted in the field of 3D modelling, and a variety of methods that allow us to reconstruct 2D images into 3D were created. Today, 3D graphics industry creates models that can no longer be distinguished from a person in the real world or on photograph. This way of modelling is also the goal of this work; to explore options to create a photorealistic 3D model shaped from the 2D images. This article has listed and briefly described methods of converting 2D images into 3D models.
\vspace{1cm}

The following objectives are aimed to be covered in this paper:

\begin{itemize}
  \item
  Working out a mathematical description of the problem
  \item
  Figure out how many views are necessary for reconstruction
  \item
  Figure out how many views are sufficient for reconstruction
  \item
  How to compute the projections from the 3D model?
  \item
  How to produce the isometric view using one or more projections?
  \item
  What interactions are necessary?
  
\end{itemize}

\\
\vspace{0.5cm}

We shall study the direction cosines and direction ratios of a line joining two points and also discuss about the equations of lines and planes in space under different conditions, angle between two lines, two planes, a line and a plane, shortest distance between two skew lines and distance of a point from a plane. Most of the above results are obtained in vector form. Nevertheless, we shall also translate these results in the Cartesian form which, at times, presents a clearer geometric and analytic picture of the situation.
\\
\vspace{0.5cm}

We shall further study the Vector Algebra and the 3D Transformations required to convert a given 3D Object to its projection view on any cross section and also do the same in a reverse manner i.e. converting the given projections back into the 3D Object by using matrices and their properties trying to exploit as many as possible and making them use to determine the number of different possible reconstructions possible (if any). We shall describe the assumptions while formulating the problem and prove to detail all the lemmas and theorems that are being used to define the model.
\\

\chapter{Vector Algebra and 3D Transformation}

\section{Homogenous coordinate System}

Three-dimensional scene description requires mainly using a 3D cartesian coordinate system. Points in space are uniquely determined by their three \textit{cartesian coordinates (x, y, z)}.

\vspace{0.5cm}
\noindent Of greater importance for computer graphics is the usage of \textit{ homogeneous} or \textit{projective coordinates }. Ordinary points in space are given four coordinates instead of three:


\subsection{Introduction}

\begin{center} 
\[ ( x , y , z ) \Leftrightarrow ( x , y , z , w ) \]
\end{center}
\\
\vspace{.3cm}
This coordinate system has wide range of applications, including computer graphics and 3D computer vision, where they allow affine transformations and projective transformations to be easily represented by a matrix
\\
\vspace{.3cm}

This introduces an obvious redundancy, so that the same point in 3D has infinitely many homogeneous coordinates, according to the equivalence
\\
\vspace{.3cm}

\begin{center} 
\[ (x, y, z, w) \equiv (x_{0}, y_{0}, z_{0}, w_{0})   \Leftrightarrow \alpha \hspace{0.2cm} (x, y, z, w) = (x_{0}, y_{0}, z_{0}, w_{0}) \]
\end{center}


\subsection{Homogenous Matrix}

\noindent The concatenation of a translation with a rotation, scaling or shear requires an awkward combination of a matrix addition and a matrix multiplication. The problem can be avoided by using an alternative coordinate system for which computations are performed by 3 × 3 matrix multiplications. Since 

\begin{center}
\[ (x' \hspace{.2cm}  y' \hspace{.2cm}  l )\hspace{.2cm} =\hspace{.2cm} (x' \hspace{.2cm}  y' \hspace{.2cm} l )    \hspace{0.2cm} \begin{pmatrix}
a & d & 0 \\
b & e & 0 \\
c & f & 1 
\end{pmatrix} \]
  
\end{center}

To this end a new coordinate system is defined in which the point with Cartesian coordinates $(x , y)$ is represented by the homogeneous or projective coordinates $( x , y , 1 )$, or any multiple $( rx , ry , r )$ with $r \neq 0 $.
\\
\vspace{.5cm}
The set of all homogeneous coordinates $(x , y , w)$ is called the \textit{projective plane} and denoted P^2 . In order to carry out transformations using matrix computations the homogeneous coordinates $( x, y, w )$ are represented by the row matrix $(x, y, w)$. 
\\
\vspace{.5cm}

\hspace{1cm} The above equation implies that any planar transformation can be performed by a $3 × 3$ matrix multiplication and using homogeneous coordinates. Sometimes homogeneous coordinates will be denoted by capitals $(X, Y, W)$ in order to distinguish them from the affine coordinates $(x, y)$.


\subsection{Key Aspects}
\begin{itemize}
  \item 
  Any point in the projective plane is represented by a triple $ (X, Y, Z)$ , called the \textit{ homogeneous coordinates}  or \textit{ projective coordinates } of the point, where $ X$ , $ Y$  and $ Z$  are not all $ 0$ .
  \item 
  The point represented by a given set of homogeneous coordinates is unchanged if \textit{  the coordinates are multiplied by a common factor }.
  \item 
  Conversely, two sets of homogeneous coordinates represent the same point if and only if one is obtained from the other by multiplying all the coordinates by the same  \textit{ non-zero constant }.
  \item 
  When $ Z$  is not $ 0$  the point represented is the point $ (X/Z, Y/Z)$  in the Euclidean plane.
  \item 
  When $ Z$  is $ 0$  the point represented is a point at $ infinity$ .
\end{itemize }

% \chapter{title}

\chapter{Assumptions}

\begin{enumerate}
    \item 
  The given Object shall consist of \textit{ only straight lines } with well defined end points and \textit{ no curved surfaces } . The same shall be assumed for the Object whose projections shall be given to us.
    \item 
  The Views that will be given to us must be \textit{ labelled with the coordinates } of their corner points and be aligned with the other necessary views in a perfect orientation for correct interpretation
\end{enumerate}


\noindent I propose to investigate aspects of Ramsey theory related to van der Waerden's theorem and to Rado's theorem. For every pair of positive integers $k$ and $r$, van der Waerden's theorem gives us the existence of a monochromatic $k$-term arithmetic progression for every $r$ colouring of the set of integers in $[1,n]$, for all sufficiently large values of $n$. The set $\{x,ax+d,bx+2d\}$ is a generalization of an arithmetic progression since $a=b=1$ makes the elements in one. Moreover, for any $a,b$, $(2a-b)x-2(ax+d)+(bx+2d)=0$, making the elements of the set fall under the category of Rado's theorem as well.  
\begin{itemize}
\item
Existence of $T(a,b;r)$ depends on $a,b$ and $r$, but is not guaranteed. One has to first to determine the degree of regularity before trying to determine $T(a,b;r)$. For $(a,b)\neq (1,1)$, it has been shown that degree of regularity is always less than or equal to $23$ by Fox and Rado\.{i}\v{c}\.{i}\'{c} {\cite{FR}}. For every pair of integers $a,b$, we propose to determine the degree of regularity, and determine or estimate $T(a,b;r)$.
\item
We would like to focus on the case where $a=b$. We know that for $r=2$, $T(a,a;2)$ exists. Whereas bounds for $T(a,a;2)$ exist, the  gap between the upper and lower bounds is quite large. The exact value where $a=b$ has been calculated upto $7$ with the help of a programme in the paper by Landman and Robertson {\cite{LR02}}. We propose to improve the upper and lower bounds, thereby decreasing the gap between them, and if possible, to find the exact value for $T(a,a;2)$. We also hope to find bounds for $T(a,a;r)$ for $r>2$.   
\item
A linear equation of the form $\sum_{i=1}^{k-1} a_ix_i=a_kx_k$ is regular if it satisfies Rado's theorem. The Rado numbers for the $2$-colour case corresponding to $k=3$ has been completely resolved. We hope to look at the $2$-colour case for $k>3$, and also hope to give some bounds for the general case with $r$ colours for $k=3$.   
\end{itemize}
%There has been a lot work done in this particular field of Ramsey Theory.Since Rado's theorem gives us a general idea about any %linear homogeneous equation, a lot of people focus on a particular equation and work upon it.In the best case scenario, exact value %is determined otherwise bounds gives us an idea about that particular Rado number.It's a hard enough problem.Exact value of Rado %numbers is known for a very few equations. We are going to work on Rado number for some general equation and hopefully get some %better results than the existing one or an exact value for that particular equation.

%\bibliography{plain}
\begin{thebibliography}{99}


\bibitem{ALM}
P. Allen, B. Landman and H. Meeks, New Bounds on van der Waerden type numbers for Generalized $3$-term Arithmetic Progressions, {\it arXiv: 1201.3842v2}


\bibitem{BL}
S. Burr and S. Loo, On Rado numbers II, unpublished.

\bibitem{EB}
E. Berlekamp, A construction for partitions which avoid long length arithmetic progressions, {\it Canadian Math Bulletin\/} {\bf 11} (1968), 409--414.


\bibitem{BB}
A. Beutelspacher and W. Brestovansky, `` Generalized Schur Numbers '', Lecture Notes in Mathematics, Springer-Verlag, Berlin, Vol. 969   pp.30--38, 1982. 

\bibitem{FLR}
N. Frantzikinakis, B. Landman and A. Robertson, On degree of regularity of generalized van der Waerden triples, {\it Advances in Applied Mathematics\/} {\bf 37} (2006), 124--128.

\bibitem{FR}
J. Fox and R. Rado\. {i}\v {c}\. {i}\' {c}, On the degree of regularity of generalized van der Waerden triples, preprint.

\bibitem{DJG}
D. J. Grynkiewicz, On some Rado numbers for generalized arithmetic progression, {\it Discrete Mathematics\/} {\bf 280} (2004), 39--50.

\bibitem{GRS90}
R. L. Graham, B. L. Rothschild and J. H. Spencer, Ramsey Theory, Second Edition, Wiley, 1990.

\bibitem{GLR}
R. L. Graham, L. Leeb and B. L. Rothschild, Ramsey Theorem for a class of categories, {\it Adv. Math.\/} {\bf 8} (1972), 417--433.

\bibitem{GS}
Song Guo and Zhi-Wei Sun, Determination of two colour Rado numbers for $\Sigma_1^m a_ix_i=x_{m+1}$, {\it Journal of Combinatorial Theory\/} {\bf 115} (2008), 345--353.

\bibitem{TG}
W. T. Gowers, A new proof of Szemer\`{e}di's theorem, {\it Geometric and Functional Analysis\/} {\bf 11} (2001), 465--588.

\bibitem{GTT}
S. Gupta, J. Thulasirangan and A. Tripathi, The two-colour Rado number for the equation $ax+by=(a+b)z$, {\it Annals of Combinatorics\/} (to appear)

\bibitem{HJ}
A. W. Hales and R. I. Jewett, Regularity and Positional Games, {\it Trans. Amer. Math. Soc.\/} {\bf 106} (1963), 222--229.

\bibitem{HS}
B. Hopkins and D. Schaal, On Rado numbers for $a_1x_1+\ldots+a_{m-1}x_{m-1}=x_m$, {\it Advances in Applied Mathematics\/} {\bf35} (2005), 433--441.

\bibitem{KS}
W. Kosek and D. Schaal, Rado number for the equation $\Sigma_1^{m-1} x_i + c=x_m$, for negative values of $c$, {\it Advances of Applied Mathematics\/} {\bf 27} (2001), 805--815.

\bibitem{LR03}
B. Landman and A. Robertson, Ramsey Theory on the Integers, Student Mathematical Library, Vol. 24, AMS, 2003. 

\bibitem{LR02}
B. Landman and A. Robertson, On generalized van der Waerden triples, {\it Discrete Mathematics\/} {\bf  256} (2002), 279--290.

\bibitem{AR}
A. Robertson, Difference Ramsey numbers and Issai numbers, {\it Advances Applied Math.\/} {\bf 25} (2000), 153--162.

\bibitem{RT}
F. P. Ramsey, On a Problem of Formal Logic, {\it Proc. London Math. Soc.\/} {\bf 30} (1930), 264--286.

\bibitem{Rado1}
R. Rado, Some recent results in combinatorial analysis, {\it Congr\`{e}s International des Mathematiciens\/}, Oslo, 1936.

\bibitem{Rado2}
R. Rado, Studien zur Kombinatorik, {\it Mathematische Zeitschrift\/} {\bf 36} (1933), 424--480.

\bibitem{Rado3}
R. Rado, Verallgemeinerung eines Satzes von van der Waerden mit Anwendungen auf ein Problem der Zahlentheorie, {\it Sonderausg. Sitzungsber. Preuss. Akad. Wiss. Phys.-Math. Klasse\/} {\bf 17} (1933), 1--10.

\bibitem{DS}
D. Schaal, On generalized Schur numbers, {\it Congressum Numerantium\/} {\bf 98} (1993), 178--187.

\bibitem{IS}
I. Schur, Uber die Kongruenz $x^m+y^m=z^m \bmod{m}$, {\it Jahresbehricht der Deutschen Mathematiker-Vereinigung\/} {\bf 25} (1916), 114--117.  

\bibitem{VVW}
B. L. van der Waerden, Beweis einer Baudetschen Vermutung, {\it Nieuw Arch. Wisk. \/} {\bf 15} (1927), 212--216.

\end{thebibliography}

\end{document}  




% \section{Schur numbers}

% \noindent Schur numbers are the least positive number $s=\tt{s(r)}$ such that for every $r$-colouring of first $s$ positive integers or $[1,s]$, we have a monochromatic solution to the equation $x+y=z$.

% \begin{thm} {\bf (I. Schur \cite{IS})}\\[5pt] 
% For any $ r\ge 1$, there exist a positive integer $\tt{s(r)}$ such that for every $r$ colouring of $[1,\tt{s(r)}]$, we have a monochromatic solution to the equation $ x+y=z$.
% \end{thm}

% \noindent A triple $\big(x, y, z \big)$ that satisfies the equation $x+y=z$ is called a Schur triple. The only known exact values for Schur numbers are:
% \[ \tt{s(1)}=2, \quad \tt{s(2)}=5, \quad \tt{s(3)}=14, \quad \tt{s(4)}=45. \]

% \noindent The colouring used in the proof of Schur's Theorem gives a bijection between edge colouring of ${\K}_n$ and colouring of $[n-1]$. The definition of this colouring implies that monochromatic triangles correspond to Schur triples. With $n={\R}_r(3)$, this gives
% \[ \frac{1}{2}(3^r+1) \le \tt s(r) \le {\R}_r(3)-1 \le 3r!-1. \]

% \noindent Let $L(t)$ represents the equation $x_1+x_2+\ldots+x_{t-1}=x_t$ where $ x_1,x_2, \ldots ,x_t$ are the unknown variables.

% \begin{thm}{\bf (A. Robertson \cite{AR})} \\[5pt]
% For $r \ge 1$ and, for $1 \le i \le r$, assume that $k_i \ge 3$. Then there exist a least positive integer $S={\S}(k_1, k_2,\ldots,k_r) $, such that for every $r$-colouring of $[1,S]$, we have a monochromatic solution to $L(k_j)$ of colour $j$ where $j \in \{1,2,\ldots,r\}$.
% \end{thm}

% \noindent The numbers $S={\S}(k_1,k_2,\ldots,k_r)$ are called {\tt generalized Schur numbers}. When $k_1=k_2=\ldots=k_r=k$, we denote it by ${\S}_r(k)$.\\

% \noindent We have a theorem that gives us the exact values for all $2$-coloured generalized Schur numbers.

% \begin{thm}
% Let $k,\ell \ge 3$. Then
% \[ {\S}(k;\ell) = \left\{ \begin{array}{ll}
%                             3\ell-4 & \mbox{for $k=3$ and $\ell$ is odd};\\
%                             3\ell-5 & \mbox{if $k=3$ and $\ell$ is even};\\
%                             k\ell-\ell-1 & \mbox{if $4 \leq k \leq \ell$}. 
%                                   \end{array}
%                 \right.
% \]  
% \end{thm}

% \noindent We have an upper and lower bound for generalized Schur number $\S_r(k)$ too.

% \begin{thm}
% Let $r \ge 2$. If $k \ge 3$, then $\S_r(k) \le \R_r(k)-1$, i.e., for every $r$-colouring of ${\K}_\R$ we have monochromatic ${\K}_k$ in some colour $j \in\{1,2,\ldots,\R\}$.
% \end{thm}

% \begin{thm}
% Let $r \ge 2$. If $ k \ge 3$, then 
% \[ \S_r(k) \ge \frac{k^{r+1}-2k^r+1}{k-1}. \]
% \end{thm}

% \noindent Now we will introduce the concept of {\tt regularity}. Let $S$ be a system of linear homogeneous equations. We say that $S$ is $r$-{\tt regular} if, for every $r$-colouring of positive integers, there is a monochromatic solution to $S$. If $S$ is $r$-regular for all $r \ge 1$, then $S$ is said to be {\tt regular}. 

% \section{Rado numbers}

% \begin{thm} {\bf (R. Rado \cite{IS})} \\ [5pt]
% Let $k \ge 2$. Let $ c_i$ be non zero integers, $1 \le i \le k$, be constants. Then 
% \[ \sum_{i=1}^k c_ix_i = 0 \] 
% is regular if and only if there exists a non-empty $D \subseteq \{c_i: 1 \le i \le k\}$ such that $\sum_{d \in D} d=0$.
% \end{thm}

% \begin{thm} {\bf (R. Rado \cite{IS})} \\ [5pt]
%  Let $\epsilon (b)$ represent the linear equation
% \[c_1x_1+c_2x_2+\ldots+c_kx_k=b,\]
% where $k\ge 2$ and each $c_i$ is a nonzero integer. Let $s=c_1+\ldots+c_k$. Then the equation $\epsilon (b)$ is regular if and only if either 
% \begin{itemize} 
% \item[{\rm (i)}]
% $\frac{b}{s}$ is a positive integer or
% \item[{\rm (ii)}]
% $\frac{b}{s}$  is a negative integer and $\epsilon(0)$ is regular.
% \end{itemize}
% \end{thm}

% \begin{thm}{\bf (Rado's ``Columns Condition")}\\ [5pt]
% Let $C=\big(\ora{c_1}, \ldots, \ora{c_n}\big)$ be a $k \times n$ matrix, where $\ora{c_i} \in \Z^k $ for $1 \le i \le n.$ We say that $C$ satisfies the``{\tt \/Columns Condition}" if we can order the columns $\ora{c_i}$ with indices $1=i_0 < i_1 < \ldots <i_s=n$ such that the following two conditions holds for $\ora{s_j}=\sum_{i_{j-1}+1}^{i_j}\ora{c_i}$ for $2 \le j \le t$.
% \begin{itemize}
% \item[{\rm (i)}]
% $\ora{s_1}=\ora 0$;
% \item[{\rm (ii)}]
% $\ora{s_j}$ can be expressed as a linear combination of $\ora{c_1}, \ldots, \ora{c_{j-1}}$ for  $2 \le j \le t$.
% \end{itemize}
% \end{thm}

% \begin{thm}{\bf (Rado's Theorem for a system of equations)}{\bf (R. Rado, \cite{Rado1,Rado2,Rado3})} \\[5pt]
% A system of linear homogeneous equations $S$ denoted by $A \ora{x}=0$ is regular if and only if $A$ satisfies the ``{\tt \/Columns Condition}". Moreover $S$ has a monochromatic solution of distinct positive integers if and only if $S$ is regular and there exist distinct (not necessarily monochromatic integers) that satisfy $S$.
% \end{thm}

% \begin{thm}{\bf  (D. Schaal \cite{DS})} \\[5pt]
% Let $b \geq 1$, $k \geq 3$, and let $\epsilon(b)$ represent the equation $x_1+\ldots+x_{k-1}-x_k=-b$. Then Rado number ${\tt{r}}(\epsilon(b);2)$ does not exist precisely when $k$ is even and $b$ is odd. Furthermore, we have
% \[{\tt{r}}(\epsilon(b);2) = k^2+(b-1)(k+1) \]
% whenever $\tt{r}(\epsilon(b);2)$ exists.
% \end{thm}

% \begin{thm}{\bf (Burr \& Loo {\cite{BL}})} \\[5pt]
% For $b \geq 1$, Rado number {\tt{r}}(x+y-z=b;2) is 
% \[ {\tt{r}}(x+y-z=b;2) = b - \left\lceil \frac{b}{5} \right\rceil + 1. \]
% \end{thm}

% \noindent Now some known Rado numbers for any given equation and the number of colours used is $2$. 

% \begin{thm}{\bf (Hopkins \& Schaal {\cite{HS}})} \\[5pt]
% Let $a_1,\ldots,a_{m-1}$ be positive integers, $m \ge 3$. Let $t=\min\{a_1,a_2,\ldots,a_{m-1}\}$ and $b=a_1+a_2+\ldots+a_{m-1}-t$. Then Rado number ${\tt{r}}(a_1x_1+\ldots+a_{m-1}x_{m-1}=x_m;2)$ is
% \[ {\tt{r}}(a_1x_1+\ldots+a_{m-1}x_{m-1}=x_m;2) \geq tb^2+(2t^2+1)b+t^3. \]
% Moreover, if $t=2$, 
% \[ {\tt{r}}(a_1x_1+\ldots+a_{m-1}x_{m-1}=x_m;2) = 2b^2+9b+8. \].
% \end{thm}

% \begin{thm}{\bf (Burr \& Loo) {\cite{BL}})} \\[5pt]
% Let $a,b \geq 1$ with $(a,b)=1$. Then Rado number in $2$ colours {\tt{r}}(ax+by=bz;2) is
% \[ {\tt{r}}(ax+by=bz;2) = \left\{ \begin{array}{ll}
%                             a^2+3a+1 & \mbox{if $b=1$};\\
%                             b^2 & \mbox{if $a<b$};\\
%                             a^2+a+1 & \mbox{if $2 \leq b \leq a$}. 
%                                         \end{array}
%                            \right.
% \]
% \end{thm}

% \begin{thm}{\bf (Grynkiewicz) {\cite{DJG}})} \\[5pt]
% Let us consider the equation $x_1+x_2-2x_3=c$ where $c$ is any integer. Now a few restraints on the given equation.
% \begin{eqnarray*}
% L_1(c) = & x_1+x_2-2x_3=c, & \\
% L_2(c) = & x_1+x_2-2x_3=c, & x_i \neq x_j \text{where} i \neq j,\\
% L_3(c) = & x_1+x_2-2x_3=c, & x_1>x_2>x_3, \\
% L_4(c) = & x_1+x_2-2x_3=c, & x_3>x_2>x_1, \\
% L_5(c) = & x_1+x_2-2x_3=c, & x_1>x_3>x_2.
% \end{eqnarray*}
% and $S_i(c)$ corresponds to $L_i(c)$. $S_i(c)$ denotes the minimum integer, if it exists, such that for every $2$ colouring from $[S_i(c)]  \rightarrow  \{0,1\}$, we have a monochromatic solution for the given equation and, otherwise $S_i(c)=\infty.$ Then
% \begin{itemize}
% \item[{\rm (i)}]
% For $i \in [5]$ and $c$ odd, $S_i(c)=\infty$.
% \item[{\rm (ii)}] 
% For $c$ even, $S_1(c)=|c|+1$.
% \item[{\rm (iii)}]
% For $c \geq 10$ and even, $S_3(c)=S_2(c)=c+4$.
% \item[{\rm (iv)}]
% For $c \leq -10$ and even, $S_2(c)=S_4(c)=-c+4$.
% \item[{\rm (v)}]
% For $c \leq 8$ and even, $S_3(c)=\infty$.
% \item[{\rm (vi)}]
% For $c \geq -8$ and even, $S_4(c)=\infty$.
% \item[{\rm (vii)}]
% For $c$ even, $S_5(c)=2|c|+10$.
% \end{itemize}
% \end{thm}

% \begin{thm} {\bf (Guo \& Sun {\cite{GS}})} \\[5pt]
% Let $a_1,\ldots,a_m$ be some positive integers and the equation under cosideration is $\sum_{i=1}^m a_ix_i=x_{m+1}$. Then the Rado number ${\tt{r}}(\sum_{i=1}^m a_ix_i=x_{m+1};2)$ is $av^2+v-a$, where $a=\min(a_1,\ldots,a_m)$ and $v=\sum_1^m a_i$. 
% \end{thm}

% \begin{thm} {\bf (Schaal {\cite{DS}})} \\[5pt]
% For every integer $m$ and $c$, let ${\tt{r}}(m,c)$ denote the $2$ colour Rado number for the equation $\sum_{i=1}^{m-1} x_i + c=x_m$. If $m$ is odd or $c \geq 0$ and even, then ${\tt{r}}(m,c)=m^2+(c-1)(m+1)$.
% \end{thm}

% \begin{thm} {\bf (Beutelspacher \& Brestovansky {\cite{BB}})} \\[5pt]
% For every integer $m$ and $c$, let ${\tt{r}}(m,c)$ denote the $2$ colour Rado number for the equation $\sum_{i=1}^{m-1} x_i + c=x_m$. For $m \geq 3$ and $c=0$, ${\tt{r}}(m,0)=m^2-m-1$.
% \end{thm}

% \begin{thm} {\bf (Kosek \& Schaal {\cite{KS}})} \\[5pt]
% For every integer $m \geq 3$ and every integer $c$, let ${\tt{r}}(m,c)$ denote the $2$ colour Rado number for the equation $\sum_{i=1}^{m-1} x_i + c=x_m$. Here $m$ is even or $c$ is odd as ${\tt{r}}(m,c)=\infty$ when $m$ is even and $c$ is odd. Then for $c<0$, we have the following cases
% \[ {\tt{r}}(m,c) = \left \{ \begin{array}{ll}
%                                        m^2-(c-1)(m+1) & \mbox{when $-m+2<c<0$}; \\
%                                        2 & \mbox{when $c=-2(m-2)$};\\
%                                        3 & \mbox{when $c=-2(m-2)-1$};\\
%                                        j(m-1)+c & \mbox{where $-j(m-2) \leq c \leq -(j-1)(m-1)$,} \\
%                                                 & \mbox{$j=3,\ldots,m-1$};\\
%                                        \big \lceil {\frac {1-(m+1)c}{m^2-m-1}} \big \rceil & \mbox{when $c<-(m-1)(m-2)$}.
%                                      \end{array}
%                         \right. 
% \]
% Moreover ${\tt{r}}(m,c) \leq m$ when $-2(m-2)+1 \leq c \leq -m+2$.

% \end{thm}

% \begin{thm}{\bf (Gupta, Thulasirangan \& Tripathi {\cite{GTT}})} \\[5pt]
% For the equation of the form $ax+by=(a+b)z$ where $a$ and $b$ are integers, Rado number ${\tt{r}}(ax+by=(a+b)z;2)$ is given by
% \[ {\tt{r}}(ax+by=(a+b)z;2) = \left\{ \begin{array}{ll}
%                             4(a+b)-1 & \mbox{if $a=1$ or $4\mid b$ or $(a,b)=(3,4)$};\\
%                             4(a+b)+1& \mbox{otherwise}.
%                                       \end{array}                                  
%                              \right.
% \]
% \end{thm}

% \begin{thm} {\bf (Burr \& Loo {\cite{BL}})} \\[5pt]
% For $a\geq 1$, 
% \[ {\tt {r}} (ax+ay=z;2) = a(4a^2+1). \]
% \end{thm}

% %\section{Proposed Research Work}

% %\noindent A linear equation of the form $\Sigma_1^{j-1}a_ix_i=a_jx_j$ is regular if it satisfies Rado's theorem. This gives us a %general idea what sort of coefficients should be there in order to have a well defined Rado number corresponding to a given equation %and for all values of $r$. One thing that we would like to work on is an equation with three variables i.e. $a_1x_1+a_2x_2=a_3x_3$ %where $a_1+a_2=a_3$. It is regular and has been calculated for $2$ colours. One can always work with $3$ colours and work on it's %Rado number. Other thing would be to generalize the case i.e. $\Sigma_1^{j-1} a_i=a_j$ with $2$ colours and improve the existing %bounds and if possible get an exact value. %There has been a lot work done in this particular field of Ramsey Theory.Since Rado's %theorem gives us a general idea about any %linear homogeneous equation, a lot of people focus on a particular equation and work upon %it.In the best case scenario, exact value %is determined otherwise bounds gives us an idea about that particular Rado number.It's a %hard enough problem.Exact value of Rado %numbers is known for a very few equations. We are going to work on Rado number for some %general equation and hopefully get some %better results than the existing one or an exact value for that particular equation.
